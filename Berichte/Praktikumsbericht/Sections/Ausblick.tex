\section{Ausblick}
\label{Ausblick}
Nachdem alle Experimente durchgeführt und evaluiert worden sind, werden die gesammelten Erfahrungen festgehalten.

\subsection{Erfolglose Versuche}
Um gute Modelle zu finden, die die Aufgabe lösen konnten, gab es einige Versuche, die zwar vielversprechend schienen, aber in der Praxis keine erfolgreichen Ergebnisse lieferten.

\subsubsection{Easy Data Augmentation}
\label{EDA}
Es wurde versucht, aus den unterrepräsentierten Sublabeln weiteren Text zu erzeugen, um ein ausgewogeneres Verhältnis beim Training zu haben. Dazu wurde Easy Data Augmentation \cite{Wei2019} verwendet. Dabei wurden sowohl Synonyme genutzt, als auch beliebige Ersetzungen, Löschungen und der Tausch von Wörtern. Bei dem ersten Testlauf wurde auf den augmentierten Daten trainiert und getestet und es kam zu einer Verbesserung bei den unterrepräsentierten Sublabeln bzgl. des Macro Average Recall. Als dann allerdings auf dem augmentierten Modell mit den Originaldaten getestet worden ist, kam es zu einer deutlichen Verschlechterung selbst im Vergleich zum Modell, welches ohne augmentierte Daten trainiert worden ist. Deshalb wurde der Weg nicht weiter verfolgt, sondern weitere echte Daten beschafft.

\subsubsection{Gram-Schmidt Verfahren zur Erstellung von Embedding}
Embeddings sind Vektordarstellungen von Wörtern. Embeddings können z.T. in der Lage dazu sein Kontext von Texten zu lernen. Allerdings müssen sie häufig trainiert werden. \cite{Almeida2019} Ein Gedanke, der bereits früh in der Entwicklung der Modelle auftrat, war es Embeddings mithilfe numerischer Verfahren nachzubauen. Diese Idee kam bereits Yang et al. \cite{Yang2019}. Die Grundidee ist es, die Sätze aufzuspannen und aufgrund dessen die Ähnlichkeit zu einander zu bestimmen. Im Rahmen des Projektpraktikums wurden daher $n$ Artikel von jeder der $m$ Kategorien verwendet, um eine Orthonormalbasis für einen $m \cdot n$-dimensionalen Raum zu schaffen. Daraufhin wurden die Skalarprodukte zwischen den restlichen Vektoren und jedem Vektor der Orthogonalbasis bestimmt. Anschließend wurde das Skalarprodukt der Vektoren der nicht aufgespannten Vektoren mit dem der orthonormierten Vektoren gebildet. Diese Skalarprodukte, waren jeweils der $x_i$ Koordinatenpunkt, wobei $i$ die Koordinate ist, die durch $n$ aufgespannt wird. Diese wurden anschließend als Eingabe der anderen Modelle verwendet. Das Problem an dieser Lösung war, dass die Resultate schlechter geworden sind, weswegen diese Lösung nicht weiter verfolgt worden ist.

\subsubsection{Klassifizierung einzelner Wörter als Kodierung}
Da die vier ersten verwendeten Ansätze zu Problemen führten, was die Multilabelklassifikation angeht und daher der Gedanke aufkam den Kontext von Wörtern miteinzubeziehen, gab es den Gedanken, bereits einzelne Wörter zu klassifizieren und anschließend so zu kodieren, dass ähnlich klassifizierte Wörter durch naheliegende Werte dargestellt werden. Dabei wurde das Alphabet als Lexikon verwendet und die Wörter mithilfe dieses Lexikons kodiert. Diese Kodierung wurde dann klassifiziert. Das Problem an diesem Versuch, war es dass bereits die Klassifizierung der einzelnen Wörter zu schlechten Resultaten geführt hat, weswegen dieser Versuch an dieser Stelle nicht weiter verfolgt worden ist. Das Verfahren funktioniert grundsätzlich wie folgt: Zunächst werden alle Artikel in ihre Worte zerteilt. Jedes der Wörter wird aufgrund seiner Angehörigkeit gelabelt. Anschließend wird geschaut, ob ein Wort mehrere Label hat. Dieses wird entsprechend separat gelabelt. Bisher wurden nur die Wörter aus den Trainingsdaten gelabelt. Diese Wörter werden in ihre Buchstaben zerlegt und vektorisiert. Anschließend werden sie verwendet, um ein Modell zu trainieren. Neue Wörter, das heißt z.B. Wörter aus den Testdaten, die nicht in den Trainingsdaten waren, werden in einzelne Buchstaben zerlegt und vektorisiert. Daraufhin werden sie klassifiziert und erhalten ihren Code, mit dem sie als Vektor dargestellt werden.

\subsection{Ausbaumöglichkeiten}
Um die Forschung auf diesem Gebiet fortzuführen, könnte man, weitere Modelle hinzufügen. Besonders interessant könnten dabei Entscheidungsbäume sein, um abzulesen welche Wörter den grössten auf die Klassifizierug haben. Eine weitere Architektur, die positive Effekte zur Bekämpfung der ungleich verteilten Label haben könnte, wäre SetFit \cite{Tunstall2022}. Ausserdem lag der Fokus während dem Praktikum eher auf dem Text der einzelnen Artikel, während die URL nicht zur Anwendung kam. Daher könnte man versuchen, in einem weiteren Schritt, eher auf dieses Attribut zu fokussieren und die Metadaten der einzelnen Seiten zur Klassifizierung zu verwenden. Dabei könnten sowohl maschinielle Methoden zur Anwendung kommen oder Algorithmen mit denen man z.B. bestimmen könnte, wie sehr die Artikel miteinander vernetzt sind.
