\section{Ausblick}
\label{Ausblick}
Nachdem alle Experimente durchgeführt und evaluiert worden sind, werden die gesammelten Erfahrungen festgehalten.

\subsection{Erfolglose Versuche}
Es wurden weitere Ansätze getestet, die jedoch keine Verbesserung erzielten und deswegen eingestellt wurden.

\subsubsection{Easy Data Augmentation}
\label{EDA}
Um das Ungleichgewicht bei unterrepräsentierten Sublabeln auszugleichen, wurde versucht, zusätzliche Texte mithilfe von Easy Data Augmentation \cite{Wei2019} zu generieren. Dabei wurden Synonymaustausch, zufällige Ersetzungen, Löschungen und Wortvertauschungen angewendet. Ein erster Testlauf  auf den augmentierten Daten zeigte eine Verbesserung  des Macro Average Recall für die unterrepräsentierten Labels. Allerdings haben die so trainierten Modelle auf den Originaldaten eine deutlich schlechtere Leistung gehabt, weswegen dieser Ansatz verworfen wurde.

\subsubsection{Gram-Schmidt Verfahren zur Erstellung von Embedding}
\label{sec:gram}
Das Gram-Schmidt-Verfahren wurde zur Erzeugung von Vektorrepräsentationen eingesetzt, indem $n$ Artikel aus $m$ Kategorien eine Orthonormalbasis für einen $m\cdot n$-dimensionalen Raum definierten. Die Skalarprodukte der verbleibenden Vektoren mit dieser Basis dienten als Koordinaten der Artikel in diesem Raum. Ein sehr ähnlicher Ansatz findet sich bei Yang et al. \cite{Yang2019}. Da die Methode zu einer Verschlechterung der Ergebnisse führte, wurde sie verworfen.

\subsubsection{Klassifizierung einzelner Wörter als Kodierung}
\label{sec:wordcoding}
Um den Kontext der Wörter miteinzubeziehen, gab es den Versuch, bereits die einzelnen Wörter zu klassifizieren und dementsprechend zu kodieren. Dazu wurden alle Artikel in ihre Wörter zerteilt. Die Wörter erhielten das Label ihres jeweiligen Artikels, wobei Wörter mit mehreren Zugehörigkeiten separate Klassen waren. Die Wörter wurden vektorisiert, indem sie auf Basis ihrer Buchstaben kodiert worden sind. Anschließend wurden sie mithilfe eines Modells, z.B. einem SVM, klassifiziert. Der Versuch wurde abgebrochen, als bereits die binäre Klassifizierung einzelner Wörter keine ausreichenden Resultate zurückgab.

\subsection{Ausbaumöglichkeiten}
Es gibt mehrere Möglichkeiten, um die Ergebnisse dieser Arbeit möglicherweise weiter zu verbessern. Zunächst könnten weitere Vorverarbeitungsmethoden getestet werden. Beispielsweise könnten andere Vektorisierungsmethoden zu einer besseren Repräsentation der Artikel führen. Eine weitere Möglichkeit wäre, weitere Modellarchitekturen zu testen. Zusätzlich könnte es hilfreich sein, Metadaten wie Autoren, Editoren oder die Anzahl der Bearbeitungen in die Analyse einzubeziehen, um zusätzliche Kontextinformationen zu nutzen.
