\subsection{Support Vector Machine}
\label{SVM}

Die Support Vector Machine (SVM) ist ein überwachtes Lernverfahren zur Klassifikation, das darauf abzielt, eine die Klassen trennende Hyperebene mit maximalem Margin zu finden. Wie zum Beispiel in \cite{Joachims1998} beschrieben, eignet sich das Verfahren besonders gut zur Textklassifizierung.

Im Rahmen des Projekts wurden verschiedene Varianten der SVM von Scikit-Learn \cite{Pedregosa2011} getestet. Da der lineare Kernel auf den vorliegenden Daten leicht bessere Ergebnisse lieferte als die Alternativen RBF, Sigmoid und Polynomial, konnte die Implementierung \texttt{LinearSVC} eingesetzt werden. Diese unterstützt ausschließlich lineare Kernel, skaliert aber besser mit der Anzahl der Wikipedia-Artikel als die Implementierung \texttt{SVC}, welche kompatibel mit weiteren Kernel-Funktionen ist.

\texttt{LinearSVC} basiert auf der Bibliothek \texttt{LIBLINEAR}, die in \cite{Fan2008} beschrieben ist und löst das Optimierungsproblem
\begin{equation*}
  \min_{w,\, b} \frac{1}{2} w^T w + C \sum_{i=1}^{l} \left( \max(0, 1 - y_i (w^T x_i + b)) \right)^2.
\end{equation*}
Dabei ist \( w \in \mathbb{R}^d \) der Gewichtsvektor, \( b \in \mathbb{R} \) der Bias-Term, \( x_i \in \mathbb{R}^d \) ein Element aus dem Trainingsdatensatz mit Label \( y_i \in \{-1, 1\} \), \( l \) die Anzahl der Trainingsbeispiele und \( C > 0 \) der Regularisierungsparameter. Einem \( x \in \mathbb{R}^d \) wird dabei die Klasse \( sign(w^T x + b) \) zugeteilt.