Im Rahmen des Kick-Offs wurde beschlossen, dass Discord (bereitgestellt über Alexander Kunze) und Github (bereitgestellt von Robin Suxdorf) als Kollaborationsplattformen dienen. Ein wöchentlicher Jour-Fixe sichert den regelmäßigen Austausch. Jeder Teilnehmer verantwortet die Weiterentwicklung seiner Methode. Das bedeutet, er entwickelt die Methode weiter, gibt zum Jour-Fixe ein Update zum Stand und teilt mit, wenn es Herausforderungen gibt. Das Team unterstützt dabei jeden Leiter und gibt Feedback bei jeder Statusvorstellung.


\subsection{Verwendete Programmiersprache}
Damit ein guter Vergleich der Methoden stattfinden konnte, musste sich das Team auf eine Programmiersprache einigen. Dabei wurde Python gewählt. Die Wahl wurde getroffen, weil Python einige Programmbibliotheken hat, die sich für die im Projektpraktikum gestellten Aufgaben, gut eignen. Die Bibliotheken, die besonders zu dieser Entscheidung beigetragen haben, waren Scikit Learn \cite{skicitLearnRef}, welches besonders bei den klassischen Verfahren Anwendung fand und Pytorch \cite{pytorchRef}, welches hauptsächlich bei den Ansätzen mit Neuronalen Netzen Anwendung fand.

\subsection{Verwendete Tools und Techniken}
Damit die Zusammenarbeit funktionieren konnte, musste sich das Team auf eine einheitliche Vorgehensweise einigen. Darüber hinaus mussten einige Schritte durchgeführt werden, die notwendig waren, um die Verfahren des maschinellen Lernens durchführen zu können. Die Aufgabenstellung ist ein Teilgebiet des Natural Language Processing. Aus diesem Grund, mussten die Daten erst so aufbereitet werden, dass der Computer mit ihnen Berechnungen machen konnte. Die Methoden, die dafür Anwendung fand werden genauer in \textbf{Referenz ergänzen} erläutert. Weitere Vorverarbeitungsschritte waren die Bereinigung der Datensätze, um sie optimal vorzubereiten. Genauere Details werden in \textbf{Referenz ergänzen}angesprochen. Für die Erweiterung des Datensatzes wurde ein weiterer Datensatz hinzugezogen. Außerdem wurden die Daten augmentiert, worüber in \textbf{referenz ergänzen} mehr ergänzt wird. Anschließend wurden die maschinellen Verfahren angewandt, die in \textbf{Referenz ergänzen} genauer beschrieben werden. Danach wurden die Daten mit verschiedenen Metriken analysiert, die in \textbf{Referenz ergänzen} genauer erläutert werden.