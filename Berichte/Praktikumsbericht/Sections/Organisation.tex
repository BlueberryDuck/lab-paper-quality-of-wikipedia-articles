Um eine kontinuierliche Zusammenarbeit zu gewährleisten, wurde im Rahmen des Kick-Offs beschlossen, dass Discord
%(bereitgestellt über Alexander Kunze)
und GitHub
%(bereitgestellt von Robin Suxdorf)
als Kollaborationsplattformen dienen. Videocalls i.d.R. alle zwei Wochen und die schriftliche Kommunikation über Discord-Nachrichten dienten dem Austausch zu inhaltlichen Projektthemen und der gegenseitigen Unterstützung. Auf GitHub wurde das Repository \cite{Suxdorf2025Quality} für die Entwicklung, Vortragsfolien und den Abschlussbericht genutzt. %Ein wöchentlicher Jour-Fixe sichert den regelmäßigen Austausch. 
%Jeder Teilnehmer verantwortet die Weiterentwicklung seiner Methode. Das bedeutet, er entwickelt die Methode weiter, gibt zum Jour-Fixe ein Update zum Stand und teilt mit, wenn es Herausforderungen gibt. Das Team unterstützt dabei jeden Leiter und gibt Feedback bei jeder Statusvorstellung.

\label{Programmiersprache}
%Damit ein guter Vergleich der Methoden stattfinden konnte, musste sich das Team auf eine Programmiersprache einigen. 
Das Team einigte sich auf die Programmiersprache Python.
%Dabei wurde Python gewählt. 
Die Wahl wurde zum einen getroffen, weil Python einige Programmbibliotheken bereitstellt, die sich für die im Projektpraktikum gestellten Aufgaben gut eignen. Die Bibliotheken, die besonders zu dieser Entscheidung beigetragen haben, sind Scikit-Learn \cite{Pedregosa2011}, welche besonders bei den klassischen Verfahren Anwendung findet, und Pytorch \cite{Ansel2024}, welche bei den Ansätzen mit neuronalen Netzen genutzt wird. Zum anderen waren im Projektteam die Kenntnisse bei Python bei allen Mitgliedern am weitesten ausgeprägt. 

Jedes Projektmitglied übernahm für ein Verfahren die Leitung. Der Umsetzungsstand und mögliche Herausforderungen wurden in den zweiwöchentlichen Terminen besprochen. Bilaterale Absprachen wurden für die Codereview und die Pipeline-Integration genutzt. 