\subsection{Vorverarbeitung und Repräsentation der Daten}
\label{sec:vorverarbeitung}

Die Wikipedia-Artikel wurden vorverarbeitet, indem Sonderzeichen und Interpunktion entfernt, alle Zeichen in Kleinbuchstaben umgewandelt und häufige Stoppwörter entfernt wurden. Zudem wurde der Porter-Stemmer \cite{Porter2006} zur Reduktion auf Wortstämme eingesetzt, während Zahlen beibehalten wurden.

Für die Vektorisierung wurden verschiedene Ansätze getestet: die klassische Count-Vektorisierung, das Bag-of-Words-Modell, sowie die semantischen Repräsentationen Word2Vec \cite{Mikolov2013} und GloVe \cite{Pennington2014}. Die besten Ergebnisse wurden mit der TF-IDF-Vektorisierung erzielt. Dabei ist die \textit{term frequency} $\tf(t, d)$ eines Terms $t$ in einem Dokument $d$ definiert als
\begin{equation*}
    \tf(t, d) = \frac{f_{t, d}}{\sum_{t'\in d}f_{t', d}},
\end{equation*}
wobei $f_{t, d}$ die absolute Häufigkeit von $t$ in $d$ ist. Die \textit{sublineare Skalierung} erfolgt durch
\begin{equation*}
    \wf(t, d) = \begin{cases}
        1 + \log \tf(t, d) & \text{falls } \tf(t, d) > 0, \\
        0                  & \text{sonst}
    \end{cases}
\end{equation*}
Die \textit{inverse document frequency} $\idf(t, D)$ ist definiert als
\begin{equation*}
    \idf(t, D) = \log\frac{\vert D\vert}{\vert\{d\in D \mid t\in d\}\vert},
\end{equation*}
wobei $D$ die gesamte Dokumentensammlung ist. Der endgültige \textit{WF-IDF}-Wert ergibt sich aus
\begin{equation*}
    \wfidf(t, d, D) = \wf(t, d)\cdot \idf(t, D).
\end{equation*}

Die Implementierung erfolgte mit \texttt{TfidfVectorizer} aus Scikit-Learn \cite{Pedregosa2011} unter Nutzung folgender Parameter: \texttt{ngram\_range: [1, 1]} (ausschließlich Einzelwörter), \texttt{max\_df: 0.9} (Ignorieren von Wörtern, die in über 90\% des Korpus vorkommen), \texttt{min\_df: 0.001} (Entfernung sehr seltener Wörter, mögliche Rechtschreibfehler), \texttt{max\_features: 10.000} (Begrenzung der Vokabulargröße zur Laufzeitverbesserung und Minimierung des Overfit-Risikos) sowie \texttt{sublinear\_tf: true} (logarithmische Skalierung der Termfrequenzen zur Abschwächung extremer Häufigkeiten).
