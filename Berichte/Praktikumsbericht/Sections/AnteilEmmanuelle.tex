Die organisatorische Unterstützung, die Emmanuelle vor allem leistete, war die Protokollführung bei den regelmäßigen Treffen. 
%Ausserdem teilte sie den Bericht in verschiedene Teildateien auf, sodass es beim parallelen Arbeiten nicht zu Problemen kam.
Beim Bericht ergänzte Emmanuelle den Abschnitt \ref{Transformer}. Sie ergänzte Verwendete Programmiersprache 3.1, 3.2 Verwendete Tools und Techniken 4.2 Probleme des ursprünglicher Datensatz und 4.3 Weitere Daten. Sie schrieb auch beim 5 Abschnitt den einleitenden Text. Sie ergänzte die Unterabschnitte von 7.1 Erfolglose Versuche.
Beim Zwischenvortrag entwickelte sie eine erste Version, die als Basis der Struktur des Vortrages dienen sollte. 
Emmanuelle implementierte den 5. Ansatz, in dem ein Transformer-Modell fine-getuned worden ist. Sie machte auch noch weitere Tests die für den 5. Ansatz gedient haben, aber nicht unbedingt im finalen Ergebnis zu sehen waren, weil sie nicht unbedingt erfolgreich waren.
Anfangs haben alle Analysen des Datensatzes durchgeführt, entsprechend auch Emmanuelle. Um einen geeigneten 5. Ansatz zu finden, machte sie weiterführende Analysen bzgl. des Datensatzes und weitere Recherchen, um Methoden zu finden, die auftretenden Probleme der ersten vier Ansätze zu beheben oder abzuschwächen. Diese Analysen führten auch zu entsprechenden Versuchen, in der Entwicklung von Ansätzen.
Außerdem machte sie Versuche, um nach weiteren Datensätzen zu suchen. %Sollen wir die Datensätze noch ergänzen
Diese Versuche waren allerdings nicht besonders erfolgreich, weil das Ziel der Recherchen entweder nicht erfüllt wurde oder die gefundenen Artikel, die Probleme des Datensatzes nicht gelöst haben. 
%Sie half auch bei der Literaturrecherche für die Datenaugmentation.

%Minimale Unterstützungen der anderen Ansätze, die sie erbrachte, waren das bereitstellen ihrer GPU für das Training des CNN und eine Rechercheunterstützung, um eine Erklärung für die sehr guten Ergebnisse des SVM-Ansatzes zu finden.