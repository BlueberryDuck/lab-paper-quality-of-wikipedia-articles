Das neuronale Netz besteht aus einer versteckten Schicht mit 512 Neuronen, wobei die ReLU-Aktivierungsfunktion verwendet wird. Zudem wird zur Regularisierung \textit{Dropout} mit einer Wahrscheinlichkeit von $p=0.5$ auf die versteckte Schicht angewendet, wodurch die Ausgabe der Neuronen mit einer Wahrscheinlichkeit von 50\% auf 0 gesetzt wird. Das Netzwerk lässt sich formal wie folgt beschreiben: $W_1\in\mathbb{R}^{512\times 10000}$, $b_1\in\mathbb{R}^{512}$, $W_2\in\mathbb{R}^{K\times 512}$ und $b_2\in\mathbb{R}^{K}$. Die Ausgabe des neuronalen Netzes ist gegeben durch
\begin{equation*}
    \hat{y} = W_2\cdot \dropout(\max(W_1x + b_1, 0)), p=0.5) + b_2\in\mathbb{R}^{K}.
\end{equation*}

Für die binäre Klassifikation ($K=2$) bzw. die Multi-Klassen-Klassifikation ($K=3$) wird das Label mit dem höchsten Wert zurückgegeben, d.h. $\arg\max (\hat{y})$.

Bei der Multi-Label-Klassifikation ($K=5$) wird die \textit{Sigmoidfunktion} $\sigma\colon t\mapsto \frac{1}{1 + e^{-t}}\in [0, 1]$ komponentenweise auf $\hat{y}$ angewendet. Ein Label wird als aktiv betrachtet, wenn der entsprechende Wert von $\hat{y}_i$ größer oder gleich 0.5 ist.

Das Optimierungsziel für die Multi-Klassen-Klassifikation ist die Minimierung der Kreuzentropie
\begin{equation*}
    L = - \sum_{i=1}^{K}y_i\log(\hat{y}_i).
\end{equation*}

Bei der Multi-Label-Klassifikation wird zunächst für ein Label $i$ die Kreuzentropie
\begin{equation*}
    l_i = y_i\log (\sigma(\hat{y}_i)) + (1 - y_i)\log (1 - \sigma(\hat{y}_i))
\end{equation*}
berechnet und anschließend der Gesamtfehler als Durchschnitt der $l_i$, d.h.
\begin{equation*}
    L = \frac{1}{K}\sum_{i=1}^{K} l_i.
\end{equation*}

Die Modelle wurden jeweils 10 Epochen mit einer Lernrate von 0.001 und einer Batchgröße von 16 unter Verwendung des Adam-Optimierers trainiert.
