%Im Rahmen dieses Projekts bearbeiten wir folgende Teilaufgaben:

%\begin{enumerate} 
%\item \textbf{Analyse des Datensatzes und Identifizierung einer geeigneten Problemstellung}: Wir untersuchen den bereitgestellten Datensatz eingehend, um ein maschinelles Lernproblem zu formulieren, das mit den vorhandenen Daten gelöst werden kann. 
%\item \textbf{Aufbereitung und Vorverarbeitung des Datensatzes}: Wir bereinigen und transformieren die Daten, um sie für die Modellierung vorzubereiten. 
%\item \textbf{Anwendung von drei klassischen Methoden des maschinellen Lernens}: Basierend auf den Inhalten der Kapitel 2 und 3 des Kurses \glqq Einführung in Maschinelles Lernen\grqq{} implementieren wir drei klassische Algorithmen, um die identifizierte Problemstellung zu adressieren. 
%\item \textbf{Anwendung eines Deep-Learning-Ansatzes}: Wir recherchieren einen geeigneten Deep-Learning-Ansatz, setzen diesen um und wenden diesen auf die Problemstellung an
%\item \textbf{Entwickeln eines eigenen Ansatzes}: Im Rahmen dieser Ausarbeitung wird ein eigener Ansatz für die Problemstellung entwickelt und beschrieben. 
%\item \textbf{Interpretation und Diskussion der Ergebnisse}: Basierend auf den bisherigen Resultaten entwickeln wir eine neue Idee für einen passenden Ansatz, beispielsweise eine neue Architektur für ein neuronales Netzwerk, die wir implementieren und anwenden. \end{enumerate}

Im Rahmen des Praktikums haben alle Teammitglieder kontinuierlich an den verschiedenen Schritten des Projektes mitgeholfen. Dazu haben alle die Beiträge zu den Präsentationen und dem Abschlussbericht beigetragen. Die folgenden Abschnitte beschreiben, was jedes Gruppenmitglied gemacht hat.


%Im Kick-Off Meeting wurde Robin Suxdorf als Teamleiter und Kommunikationskanal zu den Praktikumsbetreuern gewählt.  Für die Umsetzung der Teilaufgaben wurden jeweils verantwortliche bestimmt


%\begin{enumerate}
%    \item Klassische Methode 1: Ansatz: Bayes Leiter: Sebastian Bunge
%    \item Klassische Methode 2: Ansatz: SVM Leiter: Johannes Krämer
%    \item Klassische Methode 3: Ansatz: Logistische Regression Leiter: Alexander Kunte
%    \item Deep-Learning Methode: LSTM Transformer oder Ansatz über Embeddings; Zuerst einmal werden Embeddings angeschaut Leiter Robin Suxdorf
%    \item Eigener Ansatz: wird noch genauer angeschaut Leiter: Emmanuelle Steenhof
%\end{enumerate}
%Während des gesamten Praktikums schreibt Alexander Kunze fortlaufend den Praktikumsbericht weiter.
%-Weitere Themen: Präsentation, Vorträge, usw. 