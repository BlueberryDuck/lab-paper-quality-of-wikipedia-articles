\subsubsection{Binäre Klassifikation}
Um Artikel als \textit{„good“} oder \textit{„promotional“} zu kategorisieren, wurde eine logistische Regression mit L1-Regularisierung eingesetzt, um eine robuste Modellierung und Merkmalsselektion zu ermöglichen. Zunächst wurde der gegebene Datensatz gemäß der eingesetzten Pipeline vorverarbeitet. Zur numerischen Darstellung der Texte wurde eine \textit{TF-IDF-Vektorisierung} angewandt. Hierbei wurden verschiedene Konfigurationen getestet, darunter unterschiedliche \textit{ngram}-Bereiche und eine Begrenzung der maximalen Merkmalsanzahl. Die Daten wurden anschließend in Trainings- und Testmengen aufgeteilt.\\
Das Modell wurde mit einer logistischen Regression mit L1-Regularisierung trainiert. Um die optimale Hyperparameter-Kombination zu finden, wurde ein \textit{GridSearchCV}-Verfahren mit Kreuzvalidierung eingesetzt. Hierbei wurden verschiedene Werte für den Regularisierungsparameter \(C\) sowie unterschiedliche \textit{ngram}-Bereiche getestet. \\
Das trainierte Modell wurde mit Metriken wie \textit{Precision}, \textit{Recall} und \textit{F1-Score} bewertet. Um weitere Einsichten in die Artikelstruktur zu gewinnen, wurden zusätzliche Analysen durchgeführt, darunter eine Untersuchung der Häufigkeit von Sub-Labels sowie eine statistische Analyse der Wortanzahl in den Artikeln.\\
Die Ergebnisse zeigten, dass der Ansatz zuverlässig zwischen \textit{good}- und \textit{promotional}-Artikeln unterscheiden konnte. Zudem konnten durch die L1-Regularisierung irrelevante Merkmale entfernt werden, wodurch das Modell interpretierbarer wurde.
\subsubsection{Erweiterung durch Multilabel-Klassifikation und Datenaugmentation}
Zusätzlich zur binären Klassifikation wurde das Problem als \textit{Multilabel-Klassifikation} betrachtet. Dabei wurde jedem Artikel eine oder mehrere Kategorien aus der Menge \textit{advert}, \textit{coi}, \textit{fanpov}, \textit{pr} und \textit{resume} zugewiesen. \\
Zunächst wurde der Datensatz geladen und mittels \textit{TF-IDF-Vektorisierung} in numerische Features umgewandelt. Da ein Artikel mehrere Labels gleichzeitig besitzen kann, wurde ein \textit{One-vs-Rest}-Ansatz mit L1-regularisierter logistischer Regression verwendet. Das Modell wurde mit einer \textit{Train-Test-Split}-Aufteilung trainiert und mit \textit{Precision}, \textit{Recall} und \textit{F1-Score} bewertet.\\
Da einige Klassen unterrepräsentiert waren, wurde eine Datenaugmentation durchgeführt. Dabei wurden für Artikel mit seltenen Labels durch eine \textit{Synonym-Ersetzung} neue Sätze generiert. Hierbei wurde für zufällig ausgewählte Wörter in den Texten ein Synonym aus dem \textit{WordNet}-Lexikon ersetzt. Diese augmentierten Texte wurden in den Datensatz integriert, um die Klassifikationsleistung für unterrepräsentierte Labels zu verbessern.\\
Nach der Augmentierung wurde das Modell erneut trainiert und evaluiert. Der Vergleich der Klassifikationsberichte zeigte, dass die Verwendung der erweiterten Datenbasis zu einer Erhöhung der \textit{F1-Score} führte.\\
Die durchgeführten Schritte führten zum Teil zu einer verbesserten Klassifikationsleistung, insbesondere für seltene Labels. Das Modell wurde anschließend in die Pipeline des Projektes eingefügt.

