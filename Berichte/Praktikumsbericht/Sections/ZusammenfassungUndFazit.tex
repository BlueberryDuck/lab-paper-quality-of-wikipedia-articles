\section{Zusammenfassung und Fazit}
\label{ZusammenfassungUndFazit}
Das Ziel dieser Arbeit waren die Untersuchung von Modellen zur Analyse der Qualität von Wikipedia-Artikeln. Dabei wurde zunächst der gegebenen Datensatz analysiert und Problemstellungen erarbeitet. Dieser Datensatz wrude mithilfe des Wikipedia Dumps erweitert. Weitere Versuche der Datenaugmentierung haben stattgefunden. Es wurden im Rahmen des Projekts fünf Modelle implementiert und ausgewertet. Anschliessend wurden die Ergebnisse verglichen und ausgewertet. Artefakte, welche durch die Arbeit entstanden, waren eine Pipeline, welche die 5 Modelle implementiert und eine flexible Verwendung der drei klassischen Ansätze erlaubt, wobei verschiedene Vorverarbeitungsmethoden und Vektorisierungen verwendet werden können. Ausserdem entstand durch das Praktikum ein riesiger Wikipedia Dump Datensatz, welcher 56 GB Daten zur Verfügung stellt.
Der Schluss aus der Auswertung der Resultate war... 
\\
\\
\\
Aufbauend darauf könnte man...