\section{Zusammenfassung und Fazit}
\label{ZusammenfassungUndFazit}
Diese Arbeit untersuchte die automatisierte Klassifikation von Wikipedia-Artikeln hinsichtlich ihres \textit{promotional} (werblichen) Charakters. Dabei wurden zwei Problemstellungen adressiert: die binäre Unterscheidung zwischen werblichen und nicht-werblichen Artikeln sowie die Zuordnung der Labels bei werblichen Wikipedia-Artikeln. Zur Umsetzung wurden fünf Modelle entwickelt und evaluiert. Dabei wurde der originale Datensatz durch Wikipedia-Artikel aus dem Wikipedia-Dump erweitert. Die binäre Klassifikation lässt sich durch logistische Regression, SVM oder neuronale Netze gut lösen. Die Multi-Label-Klassifikation erreichte geringere Werte. Das beste Modell für dieses Problem war der Naive Bayes-Klassifikator mit einem Recall von 66\%. Die Ergebnisse zeigen, dass klassische Methoden des maschinellen Lernens sich für die Erkennung werblicher Wikipedia-Artikel gut eignen. Die Multi-Label-Klassifikation benötigt jedoch noch weitere Verbesserungen, zum Beispiel durch eine bessere Vorverarbeitung oder fortgeschrittenere Methoden.
