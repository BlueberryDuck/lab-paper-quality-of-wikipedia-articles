\subsection{Bayes-Klassifikator}
\label{sec:bayes-klassifikator}

Der Naive-Bayes-Klassifikator ist ein probabilistisches Modell, das auf dem Bayes’schen Theorem basiert und unter der Annahme bedingter Unabhängigkeit der Merkmale Klassen zuweist \cite{Bishop2019}. In der Textklassifikation findet der Multinomial-Naive-Bayes-Klassifikator Anwendung, bei dem ein Dokument $d$ der Klasse
\begin{equation*}
    \hat{c} \;=\; \arg\max_{c \in C}\; P(c) \prod_{i=1}^{M} P(w_i \mid c)^{\tf(w_i,d)}
\end{equation*}
zugewiesen wird \cite{Manning2009}. Dabei bezeichnet $M$ die Größe des Vokabulars, $P(c)$ die a-priori Wahrscheinlichkeit der Klasse $c$ und $P(w_i \mid c)$ die bedingte Wahrscheinlichkeit des Terms $w_i$ in der Klasse $c$.

Die praktische Umsetzung erfolgte über den \texttt{MultinomialNB} von Scikit-Learn \cite{Pedregosa2011}. Dabei wurden folgende Parameter optimiert: \texttt{alpha} (zur Laplace-Glättung, um bei nicht beobachteten Merkmalen Null-Wahrscheinlichkeiten zu vermeiden) sowie \texttt{fit\_prior} (zur Schätzung der apriorischen Klassenverteilung ohne Merkmalsbetrachtung, andernfalls wird eine gleichmäßige Verteilung angenommen).
