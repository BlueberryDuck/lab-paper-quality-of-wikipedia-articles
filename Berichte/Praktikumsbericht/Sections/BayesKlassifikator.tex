\subsection{Bayes-Klassifikator}
\label{sec:bayes-klassifikator}

Der Naive-Bayes-Klassifikator ist ein probabilistisches Modell, das auf dem Bayes'schem Theorem basiert und unter der Annahme der bedingten Unabhängigkeit der Merkmale Klassen-Wahrscheinlichkeiten schätzt \cite{Bishop2019}.

Es wurde der Multinomial Naive Bayes-Klassifikator \texttt{MultinomialNB} von Scikit-Learn \cite{Pedregosa2011} eingesetzt, da er sich besonders gut für Textklassifizierungsaufgaben eignet. Dabei werden Wortfrequenzen als diskrete Zähldaten mit einer linearen Zeitkomplexität \cite{Manning2009} verarbeitet, welcher trotz der hohen Datenmenge und der hohen Dimensionalität schnell gute Ergebnisse liefert. Zur Optimierung der Klassifikation werden die Parameter \texttt{alpha} und \texttt{fit\_prior} angepasst. Mit \texttt{alpha} wird mittels Laplace-Glättung sichergestellt, dass auch bei nicht beobachteten Merkmalen keine Null-Wahrscheinlichkeiten entstehen, indem sie durch eine kleine positive Konstante ersetzt werden, während \texttt{fit\_prior} es ermöglicht, die apriorische Klassenverteilung direkt aus den Trainingsdaten zu schätzen bevor Features betrachtet werden, andernfalls wird eine gleiche Klassenwahrscheinlichkeit angenommen.
