Der Naive Bayes Klassifikator ist ein probabilistisches Modell, das auf dem Bayes-Theorem basiert und häufig für Textklassifizierungsaufgaben verwendet wird \cite{maron_1961}. Trotz der Annahme, dass die Merkmale unabhängig voneinander sind - was in der Praxis nicht immer zutrifft - liefert der Klassifikator gute Ergebnisse.

Für die Experimente wurde hauptsächlich der Multinomial Naive Bayes Klassifikator eingesetzt, da er sich besonders gut für die Verarbeitung von Textdaten eignet \cite{eyheramendy_2003}. Um optimale Parameter zu finden, kam GridSearchCV zur Hyperparameter-Optimierung zum Einsatz. Über eine Konfigurationsdatei können Hyperparameter wie Alpha und Fit-Prior gesetzt werden, wobei Grid Search ebenfalls über diese Konfiguration aktivierbar ist. Für Multilabel-Klassifikationen wurde der OneVsRestClassifier gewählt, um mehrere Labels gleichzeitig vorhersagen zu können - ergänzt durch einen optionalen RandomOverSampler, um ungleiche Labelverteilungen auszugleichen.
