\documentclass[researchlab,palatino]{AIGpaper}
% Please read the README.md file for additional information on the parameters and overall usage of AIGpaper

%%%% Package Imports %%%%%%%%%%%%%%%%%%%%%%%%%%%%%%%%%%%%%%%%%%%%%%%%%%%%%%%%%%%%%%%%%
\usepackage{graphicx}					    % enhanced support for graphics
\usepackage{tabularx}				      	% more flexible tabular
\usepackage{amsfonts}					    % math fonts
\usepackage{amssymb}					    % math symbols
\usepackage{amsmath}                        % overall enhancements to math environment
\usepackage{amsthm}                         % Nutzung von Definition
\usepackage{hyperref}                       % Zeilenumbruch in URL
\usepackage{xurl}                           % Nach hyperref laden

%%%% optional packages
\usepackage{tikz}                           % creating graphs and other structures
\usetikzlibrary{arrows,positioning}
\tikzset{
    %Define standard arrow tip
    >=stealth',
    %Define style for argument
    args/.style={circle, minimum size=0.9cm,draw=black, thick,fill=white},
}


%%%% Author and Title Information %%%%%%%%%%%%%%%%%%%%%%%%%%%%%%%%%%%%%%%%%%%%%%%%%%%
\author{Robin Suxdorf \and Sebastian Bunge \and Johannes Krämer \and Emmanuelle Steenhof \and Alexander Kunze}

\title{Web Science - Die Qualität von Wikipedia-Artikeln}


%%%% Abstract %%%%%%%%%%%%%%%%%%%%%%%%%%%%%%%%%%%%%%%%%%%%%%%%%%%%%%%%%%%%%%%%%%%%%%

\germanabstract{
Erstmal nur ein Draft. Inhalte werden weiter abgestimmt. 
}

% use this if the document is written in english
%\englishabstract{}


\begin{document}

\maketitle % prints title and author information, as well as the abstract 


% ===================== Beginning of the actual text section =====================

\section{Einleitung}



Das Ziel dieses Projektpraktikums ist die praktische Anwendung von Methoden des maschinellen Lernens auf einen vorgegebenen Datensatz aus dem Bereich der Web Science. Wir haben uns für das Thema \textbf{Qualität von Wikipedia-Artikeln} entschieden und nutzen dafür den Datensatz von Kaggle: \url{https://www.kaggle.com/datasets/urbanbricks/wikipedia-promotional-articles}\\
Im Rahmen dieses Projekts bearbeiten wir folgende Teilaufgaben:

\begin{enumerate} 
\item \textbf{Analyse des Datensatzes und Identifizierung einer geeigneten Problemstellung}: Wir untersuchen den bereitgestellten Datensatz eingehend, um ein maschinelles Lernproblem zu formulieren, das mit den vorhandenen Daten gelöst werden kann. 
\item \textbf{Aufbereitung und Vorverarbeitung des Datensatzes}: Wir bereinigen und transformieren die Daten, um sie für die Modellierung vorzubereiten. 
\item \textbf{Anwendung von drei klassischen Methoden des maschinellen Lernens}: Basierend auf den Inhalten der Kapitel 2 und 3 des Kurses \glqq Einführung in Maschinelles Lernen\grqq{} implementieren wir drei klassische Algorithmen, um die identifizierte Problemstellung zu adressieren. 
\item \textbf{Anwendung eines Deep-Learning-Ansatzes}: Wir recherchieren einen geeigneten Deep-Learning-Ansatz, setzen diesen um und wenden diesen auf die Problemstellung an
\item \textbf{Entwickeln eines eigenen Ansatzes}: Im Rahmen dieser Ausarbeitung wird ein eigener Ansatz für die Problemstellung entwickelt und beschrieben. 
\item \textbf{Interpretation und Diskussion der Ergebnisse}: Basierend auf den bisherigen Resultaten entwickeln wir eine neue Idee für einen passenden Ansatz, beispielsweise eine neue Architektur für ein neuronales Netzwerk, die wir implementieren und anwenden. \end{enumerate}

\section{Aufgabenverteilung}
\label{Aufgabenverteilung}
Im Rahmen dieses Projekts bearbeiten wir folgende Teilaufgaben:

\begin{enumerate} 
\item \textbf{Analyse des Datensatzes und Identifizierung einer geeigneten Problemstellung}: Wir untersuchen den bereitgestellten Datensatz eingehend, um ein maschinelles Lernproblem zu formulieren, das mit den vorhandenen Daten gelöst werden kann. 
\item \textbf{Aufbereitung und Vorverarbeitung des Datensatzes}: Wir bereinigen und transformieren die Daten, um sie für die Modellierung vorzubereiten. 
\item \textbf{Anwendung von drei klassischen Methoden des maschinellen Lernens}: Basierend auf den Inhalten der Kapitel 2 und 3 des Kurses \glqq Einführung in Maschinelles Lernen\grqq{} implementieren wir drei klassische Algorithmen, um die identifizierte Problemstellung zu adressieren. 
\item \textbf{Anwendung eines Deep-Learning-Ansatzes}: Wir recherchieren einen geeigneten Deep-Learning-Ansatz, setzen diesen um und wenden diesen auf die Problemstellung an
\item \textbf{Entwickeln eines eigenen Ansatzes}: Im Rahmen dieser Ausarbeitung wird ein eigener Ansatz für die Problemstellung entwickelt und beschrieben. 
\item \textbf{Interpretation und Diskussion der Ergebnisse}: Basierend auf den bisherigen Resultaten entwickeln wir eine neue Idee für einen passenden Ansatz, beispielsweise eine neue Architektur für ein neuronales Netzwerk, die wir implementieren und anwenden. \end{enumerate}

Diese Teilaufgaben wurden auf die verschiedenen Gruppenmitglieder aufgeteilt. Wie folgenden Abschnitte beschreiben, was jedes Gruppenmitglied gemacht hat.


%Im Kick-Off Meeting wurde Robin Suxdorf als Teamleiter und Kommunikationskanal zu den Praktikumsbetreuern gewählt.  Für die Umsetzung der Teilaufgaben wurden jeweils verantwortliche bestimmt


%\begin{enumerate}
%    \item Klassische Methode 1: Ansatz: Bayes Leiter: Sebastian Bunge
%    \item Klassische Methode 2: Ansatz: SVM Leiter: Johannes Krämer
%    \item Klassische Methode 3: Ansatz: Logistische Regression Leiter: Alexander Kunte
%    \item Deep-Learning Methode: LSTM Transformer oder Ansatz über Embeddings; Zuerst einmal werden Embeddings angeschaut Leiter Robin Suxdorf
%    \item Eigener Ansatz: wird noch genauer angeschaut Leiter: Emmanuelle Steenhof
%\end{enumerate}
%Während des gesamten Praktikums schreibt Alexander Kunze fortlaufend den Praktikumsbericht weiter.
%-Weitere Themen: Präsentation, Vorträge, usw. 

\subsection{Sebastian}
\subsection{Sebastian Bunge}
\label{sec:sebastian}

Sebastian Bunge f\"uhrte die Recherche zum Bayes-Klassifikator und zu Vectorizern durch, auf deren Basis er eine modulare Projektstruktur - die sogenannte "Pipeline"\ - konzipierte und implementierte. Er realisierte zentrale Module f\"ur die Datenvorverarbeitung, Feature-Extraktion mittels Vektorraumrepr\"asentation sowie Evaluationsmethoden und entwickelte den Bayes-Klassifikator. Zudem trug er zur Code-Dokumentation (\texttt{README.md}) bei, unterst\"utzte das Team bei der Integration weiterer Ans\"atze in die Pipeline und half bei Merge-Konflikten. Seine vorbereiteten Folien zu "Datenvorverarbeitung"\ und "Bayes-Klassifikator"\ sowie seine Beitr\"age in den Berichtsteilen \ref{sec:sebastian}, \ref{sec:vorverarbeitung}, \ref{sec:bayes-klassifikator}, und \ref{sec:projektstruktur} dokumentieren seinen Anteil am Projektpraktikum.


\subsection{Johannes}
Johannes Krämer hat den Konverter für den Dump der englischsprachigen Wikipedia erstellt (\ref{WPDump}) und Anpassungen an der Pipeline vorgenommen um den konvertierten Dump für das Training und die Evaluation der Modelle heranziehen zu können. Er hat den SVM-Klassifikator entwickelt und in die Pipeline integriert (\ref{SVM}). Er hat Folien für die Zwischenpräsentation beigetragen, Teile des Abschlussberichts geschrieben und gemeinsam mit Emmanuelle Steenhof die Abschlusspräsentation gehalten.

\subsection{Alexander}
Hier muss Alexander schreiben, was er gemacht hat.

\subsection{Robin}
Im Bericht ergänzte Robin die Abschnitte \ref{KNN}, \ref{sec:evaluationsmetriken} und \ref{Ergebnisse}.
Er implementierte sowohl die abstrakte Klasse für Modelle, welche als Basis der vier ersten Ansätze diente, und das künstliche neuronale Netz \ref{KNN}. Als Teamleiter war er für die Kommunikation mit der Praktikumsbetreuung und die Leitung der Meetings \ref{Organisation} zuständig. Er präsentierte den Zwischenvortrag.

%Robin war als Teamleiter tätig und übernahm die Leitung der Meetings sowie die Kommunikation mit den Praktikumsbetreuern {\ref{Organisation}}. Er unterstützte die Strukturierung des Projekts durch die Einführung einer abstrakten Klasse für Modelle, die als Grundlage für alle Ansätze diente {\ref{sec:vorverarbeitung}}. Zudem entwickelte er den Deep-Learning-Ansatz {\ref{KNN}} und erarbeitete den Zwischenvortrag, den er anschließend präsentierte.


\subsection{Emmanuelle}
Die organisatorische Unterstützung, die Emmanuelle vor allem leistete, war die Protokollführung bei den regelmässigen Treffen. 
%Ausserdem teilte sie den Bericht in verschiedene Teildateien auf, sodass es beim parallelen Arbeiten nicht zu Problemen kam.
Beim Bericht ergänzte Emmanuelle den Abschnitt \ref{Transformer}. Sie ergänzte Verwendete Programmiersprache 3.1, 3.2 Verwendete Tools und Techniken 4.2 Probleme des ursprünglicher Datensatz und 4.3 Weitere Daten. Sie schrieb auch beim 5 Abschnitt den einleitenden Text. Sie ergänzte die Unterabschnitte von 7.1 Erfolglose Versuche.
Beim Zwischenvortrag entwickelte sie eine erste Version, die als Basis der Struktur des Vortrages dienen sollte. 
Emmanuelle implementierte den 5. Ansatz, in dem ein Transformer-Modell fine-getuned worden ist. Sie machte auch noch weitere Tests die für den 5. Ansatz gedient haben, aber nicht umbedingt im finalen Ergebnis zu sehen waren, weil sie nicht umbedingt erfolgreich waren.
Anfangs haben alle Analysen des Datensatzes durchgeführt, entsprechend auch Emmanuelle. Um einen geeigneten 5. Ansatz zu finden, machte sie weiterführende Analysen bzgl. des Datensatzes und weitere Recherchen, um Methoden zu finden, die auftretenden Probleme der ersten vier Ansätze zu beheben oder abzuschwächen. Diese Analysen führten auch zu entsprechenden Versuchen, in der Entwicklung von Ansätzen.
Ausserdem machte sie Versuche, um nach weiteren Datensätzen zu suchen. %Sollen wir die Datensätze noch ergänzen
Diese Versuche waren allerdings nicht besonders erfolgreich, weil das Ziel der Recherchen entweder nicht erfüllt wurde oder die gefundenen Artikel, die Probleme des Datensatzes nicht gelöst haben. 
%Sie half auch bei der Literaturrecherche für die Datenaugmentation.

%Minimale Unterstützungen der anderen Ansätze, die sie erbrachte, waren das bereitstellen ihrer GPU für das Training des CNN und eine Rechercheunterstützung, um eine Erklärung für die sehr guten Ergebnisse des SVM-Ansatzes zu finden.

\section{Teaminterne Organisation}
\label{Organisation}
%Ich würde noch anmerken dass jedes Thema einen Leiter gekriegt hat.
Im Rahmen des Kick-Offs wurde beschlossen, dass Discord (bereitgestellt über Alexander Kunze und Github (bereitgestellt von Robin Suxdorf) als Kollaborationsplattformen dienen. Ein wöchentlicher Jour-Fixe sichert den regelmäßigen Austausch. Jeder Teilnehmer verantwortet die Weiterentwicklung seiner Methode. Das bedeutet, er entwickelt die Methode weiter, gibt zum Jour-Fixe ein Update zum Stand und teilt mit, wenn es Herausforderungen gibt. Das Team unterstützt dabei jeden Leiter und gibt Feedback bei jeder Statusvorstellung.


\subsection{Verwendete Programmiersprache}
Damit ein guter Vergleich der Methoden stattfinden konnte, musste sich das Team auf eine Programmiersprache einigen. Dabei wurde Python gewählt. Die Wahl wurde getroffen, weil Python einige Programmbibliotheken hat, die sich für die im Projektpraktikum gestellten Aufgaben, gut eignen. Die Bibliotheken, die besonders zu dieser Entscheidung beigetragen haben, waren Scikit Learn \cite{skicitLearnRef}, welches besonders bei den klassischen Verfahren anwendung fand und Pytorch \cite{pytorchRef}, welches hauptsächlich bei den Ansätzen mit Neuronalen Netzen anwendung fand.

\subsection{Verwendete Tools und Techniken}
Damit die Zusammenarbeit funktionieren konnte, musste sich das Team auf eine einheitliche Vorgehensweise einigen. Darüber hinaus mussten einige Schritte durgeführt werden, die notwendig waren, um die Verfahren des maschiniellen Lernens durchführen zu können. Die Aufgabenstellung ist ein Teilgebiet des Natural Language Processing. Aus diesem Grund, mussten die Daten erst so aufbereitet werden, dass der Computer mit ihnen Berechnungen machen konnte. Die Methoden, die dafür Anwendung fand werden genauer in \textbf{Referenz ergänzen} erläutert. Weitere Vorverarbeitungsschritte waren die Bereinigung der Datensätze, um sie optimal vorzubereiten. Genauere Details werden in \textbf{Referenz ergänzen}angesprochen. Für die Erweiterung des Datensatzes wurde ein weiterer Datensatz hinzugezogen. Ausserdem wurden die Daten augmentiert, worüber in \textbf{referenz ergänzen} mehr ergänzt wird. Anschliessend wurden die maschiniellen Verfahren angewandt, die in \textbf{Referenz ergänzen} genauer beschrieben werden. Danach wurden die Daten mit verschiedenen Metriken analysiert, die in \textbf{Referenz ergänzen} genauer erläutert werden.


\section{Datensatz und Problemstellung}
\label{Datensatz}

Bevor die Modellbildung starten konnte, mussten zunächst die Bedingungen geklärt werden. Das bedeutet, dass der Datensatz analysiert werden musste und eine entsprechende Problemstellung identifiziert werden musste. Im Folgenden wird zunächst der ursprüngliche Datensatz beschrieben und anschliessend erklärt, wie dieser in Laufe des Projekts ausgebaut und ergänzt wird. Danach wird die daraus abgeleitete Problemstellung erläutert.

\subsection{Ursprünglicher Datensatz}
%Erter Satz geändert
Als Basis wurde ein Datensatz verwendet, welcher  verschiedene Wikipedia-Artikel enthält \cite{UrsprungDatensatz}. Der Datensatz wurde in zwei Dateien unterteilt. Die erste Datei enthielt \emph{promotional} (also werbend) Artikel und wurde noch weiter unterteilt. Dabei sind folgende Label vergeben:
\begin{itemize}
    \item advert – „Dieser Artikel enthält Inhalte, die wie eine Werbeanzeige verfasst sind.“
\item coi – „Ein Hauptautor dieses Artikels scheint eine enge Verbindung zu seinem Thema zu haben.“
\item fanpov – „Dieser Artikel ist möglicherweise aus der Sicht eines Fans geschrieben, statt aus einer neutralen Perspektive.“
\item pr – „Dieser Artikel liest sich wie eine Pressemitteilung oder ein Nachrichtenartikel oder basiert weitgehend auf routinemäßiger Berichterstattung oder Sensationslust.“
\item resume – „Dieser biografische Artikel ist wie ein Lebenslauf geschrieben.“
\end{itemize}
Der zweite Datensatz enthält Wikipedia Artikel die \emph{nicht-promotional} klassifiziert sind.
\\ \\
Der zweite Datensatz enthielt Artikel, die als Good klassifiziert worden sind. Die beiden Datensätze wurden zusammen verwendet, um die gesamte Datenbasis zu bilden. Insgesamt ergab das \textbf{Anzahl Datensätze einfügen} Daten. Die Label dieser Daten war wie folgt verteilt:


\textbf{2 Bilder einfügen: 1. Bild einzelne Label, 2. Bild Kombination der Label}
\\



\subsection{Probleme des ursprünglichen Datensatzes}
Wie man anhand des Diagramms \textbf{Auf Bild referenzieren} sehen kann, sind die Daten ungleich verteilt. Daten mit dem Label \textit{good} nehmen laut \textbf{Quelle einfügen} nur 0.59\% aller Wikipediaartikel ein. Allerdings sieht man anhand der Grafik, dass sie im Vergleich zu \textbf{Prozentsatz berechnen}. Zum einen führt das zu einem Verhältnis, das nicht der Realität entspricht, zum anderen kann das die Ergebnisse der trainierten Modelle verschlechtern.



\subsection{Weitere Daten}
Um ein gutes Modell zu erstellen, welches auf Maschinellem Lernen basiert, braucht man entsprechende Datensätze. Wie in Abschnitt \textbf{Referenz einfügen} besprochen worden ist, ist der ursprüngliche Datensatz nicht ausreichend, um entsprechende Modelle zu trainieren. Aus diesem Grund wurden verschiedene Methoden ausprobiert und verwendet, um den ursprünglichen Datensatz zu erweitern.

\subsubsection{Datensatzerweiterung durch Wikipedia-Dump}
\label{WPDump}
Die naheliegendste Methode zur Erweiterung eines Datensatzes, ist das Hinzuziehen neuer Daten. Aus diesem Grund wurde der Wikipedia-Dump \textbf{Quelle hinzufügen} hinzugezogen. Dieser Datensatz ist ein offizieller Datensatz von Wikipedia, in dem alle Artikel enthalten sind. Der Dump ist für die verschiedenen Sprachversionen und in Varianten mit oder ohne Historie verfügbar. Hier wurde der Dump der englischsprachigen Wikipedia ohne Historie verwendet, der unter \url{https://dumps.wikimedia.org/enwiki/20241220/} zu finden ist. Er besteht aus zwei Dateien:
\begin{itemize}
    \item \emph{enwiki-20241020-pages-articles-multistream.xml.bz2} (circa 22 GB komprimiert und 97 GB entpackt): Enthält eine komprimierte XML-Datei mit allen 24.091.931 Seiten und den dazugehörigen Metadaten.
    \item \emph{enwiki-20241020-pages-articles-multistream-index.txt.bz2} (circa 250 MB komprimiert und 1 GB entpackt): Enthält eine Index-Datei mit 240953 Offsets in der komprimierten XML-Datei zwischen denen jeweils 100 Seiten liegen.
\end{itemize}
Mithilfe eines Python-Programms wurden alle Artikel aus dem Dump verarbeitet und in drei Kategorien aufgeteilt: \emph{good}, \emph{promo} und \emph{neutral}. Dabei sind die ersten beiden wie zuvor im Kaggle-Datensatz zu verstehen und \emph{neutral} enthält alle Artikel, die in keine der beiden anderen Kategorien fallen. Außerdem wurden alle Seiten ausgeschlossen, die keine Artikel darstellen, darunter Begriffsklärungsseiten, Umleitungen, Kategorien, Benutzerseiten und weitere Namensräume. Aufgrund der Größe des Dumps konnte dieser nicht komplett in den Arbeitsspeicher geladen werden. Unter Nutzung der in der Index-Datei gespeicherten Offsets wurden daher jeweils einzelne Abschnitte von 100 Seiten entpackt und verarbeitet. Gute Artikel wurden anhand der darin vorhandenen Templates \textit{\{\{\{good article\}\}} und \textit{\{\{\{featured article\}\}} erkannt, bei den werbenden Artikeln waren inklusive Aliasen 21 Templates zu identifizieren. Die erkannten Templates sowie alle Zeilenumbrüche wurden aus dem Artikeltext entfernt, ansonsten wurde er unverändert übernommen. Anhand von Stichproben wurde manuell verifiziert, dass die Artikel korrekt kategorisiert wurden. Insgesamt ergab sich die folgende Aufteilung:
\begin{itemize}
    \item \emph{good}: 46.882
    \item \emph{promo}: 32.633
    \item \emph{neutral}: 6.611.303
    \item \emph{skipped}: 17.401.113
\end{itemize}
Die kategorisierten Artikel wurden ähnlich dem Kaggle-Datensatz in CSV-Dateien geschrieben, deren Zeilen neben dem Text die Seiten-ID und den Titel enthalten sowie bei \emph{promo} noch die oben genannten Label. Da die Kategorien extrem ungleich verteilt sind (neutrale Artikel etwa um einen Faktor 200 häufiger als werbende), wurde anschließend mittels \textit{Reservoir Sampling} noch eine zufällige Auswahl der Artikel in weitere CSV-Dateien geschrieben, um jeweils gleich viele Samples der drei Kategorien zu erhalten. Auf dieser Auswahl konnten dann wie beim Kaggle-Datensatz die verschiedenen Lernverfahren trainiert werden.

\subsubsection{Augmentierung der Daten}
Da es nicht sicher ist, ob der Wikimedia Dump das Problem vollkommen lösen kann, wurde neben dem Hinzuziehen weiterer Daten auch versucht die Daten zu Augmentieren. Dabei sollten besonders untervetretene Klassen mehr Repräsentanten kriegen. Dabei wurden verschiedene Methoden ausprobiert, um die Daten zu augmentieren. Diese Methoden werden in \textbf{Referenz ergänzen} vorgestellt.

\subsection{Problemdefinition}
Das Ziel dieses Projekts ist die Entwicklung von Modellen zur automatisierten Klassifikation von Wikipedia-Artikeln als \emph{promotional} (werblich) oder \emph{nicht-promotional}. Dabei wird ebenfalls klassifiziert, wie ein Artikel promotional ist, also z.B. ob er eine Werbung, ein PR-Artikel usw. ist. Wikipedia strebt nach objektiven und neutralen Inhalten; daher ist die Identifizierung von Artikeln mit werbenden Charakter von großer Bedeutung, um die sachliche Qualität der Plattform zu gewährleisten.

\subsection{Zielsetzung}

Die Hauptziele des Projekts sind:

\begin{itemize} \item Entwicklung von drei klassischen maschinellen Lernmodellen und einem Deep-Learning-Modell zur Klassifikation von Wikipedia-Artikeln. \item Vergleich der Modelle anhand von Leistungsmetriken wie Genauigkeit, Präzision, Recall und F1-Score. \item Identifikation des Modells mit der besten Leistung für die gegebene Aufgabe. \end{itemize}



\section{Ansätze}
\label{Ansätze}
%https://shelf.io/blog/18-effective-nlp-algorithms-you-need-to-know/
Nachdem die passende Problemstellung festgestellt worden ist, war die nächste Aufgabe die passenden Ansätze zu auszuwählen. Dabei wurden im Rahmen des Projektpraktikum drei klassische Lernverfahren verwendet und ein Deep Learning Ansatz. Aufbauend auf diesen wurde ein 5. Ansatz konzipiert, der als Ziel hatte, die Schwächen der vorherigen Ansätze zu beheben. Im folgenden werden diese 5 Ansätze vorgestellt und anschliessend wird erklärt welche weiteren Techniken, besonders in Bezug auf Datenvorarbeitung und Datensatzergänzung verwendet worden sind.

\subsection{Bayes-Klassifikator}
\label{Bayes-Klassifikator}
Bayes

\subsection{Support Vector Machine}
\label{SVM}
\subsection{Support Vector Machine}
\label{SVM}

Die Support Vector Machine (SVM) ist ein überwachtes Lernverfahren zur Klassifikation, das darauf abzielt, eine die Klassen trennende Hyperebene mit maximalem Margin zu finden. Wie zum Beispiel in \cite{Joachims1998} beschrieben, eignet sich das Verfahren besonders gut zur Textklassifizierung.

Im Rahmen des Projekts wurden verschiedene Varianten der SVM von Scikit-Learn \cite{Pedregosa2011} getestet. Da der lineare Kernel auf den vorliegenden Daten leicht bessere Ergebnisse lieferte als die Alternativen RBF, Sigmoid und Polynomial, konnte die Implementierung \texttt{LinearSVC} eingesetzt werden. Diese unterstützt ausschließlich lineare Kernel, skaliert aber besser mit der Anzahl Wikipedia-Artikel als die Implementierung \texttt{SVC}, welche kompatibel mit weiteren Kernel-Funktionen ist.

\texttt{LinearSVC} basiert auf der Bibliothek \texttt{LIBLINEAR}, die in \cite{Fan2008} beschrieben ist und löst das Optimierungsproblem

\begin{equation*}
  \min_{w,\, b} \frac{1}{2} w^T w + C \sum_{i=1}^{l} \left( \max(0, 1 - y_i (w^T x_i + b)) \right)^2.
\end{equation*}

Dabei ist \( w \in \mathbb{R}^d \) der Gewichtsvektor, \( b \in \mathbb{R} \) der Bias-Term, \( x_i \in \mathbb{R}^d \) ein Element aus dem Trainingsdatensatz mit Label \( y_i \in \{-1, 1\} \), \( l \) die Anzahl der Trainingsbeispiele und \( C > 0 \) der Regularisierungsparameter.

Zur binären Klassifikation eines \( x \in \mathbb{R}^d \) berechnet \texttt{LinearSVC} dann \( sign(w^T x + b) \) als Klassenzugehörigkeit. Bei mehr als zwei Klassen wird das One-vs-Rest-Verfahren verwendet, also für jede Klasse ein binärer Klassifikator trainiert, der zwischen dieser Klasse und allen anderen unterscheidet. Beim Vorhersagen wird der Klassifikator mit dem höchsten Funktionswert \( w^T x + b \) ausgewählt, um die finale Klassenzuweisung zu bestimmen.
 



\subsection{Logistische Regression}
\label{Logistische Regression}
\subsection{Logistische Regression}
\label{sec:logreg}

Für die Klassifikation von Labeln wurde die logistische Regression ausgewählt, weil dies insbesondere schnell und einfach zu implementieren war. Um Overfitting zu verhindern, wurden Lasso und Ridge-Regularisierung eingesetzt. Außerdem wurden zwei Solver evaluiert: Der \texttt{liblinear}-Solver wurde als Standardmethode eingesetzt, da er sowohl \texttt{L1}- als auch \texttt{L2}-Regularisierung unterstützt. Zusätzlich wurde \texttt{saga}, eine Erweiterung des Stochastic Average Gradient, getestet, die speziell für große Datensätze geeignet ist und ebenfalls mit beiden Regularisierungen kompatibel ist.

Für die Klassifizierung bei der logistischen Regression wird das Optimierungsproblem $\min_\theta L(D,h_\theta)$ gelöst. Dabei ist D der gegebene Datensatz und $\theta$ die Parameter, über die optimiert wird. Außerdem sei $x \in D$, dann ist $h_\theta$ als Sigmoidfunktion definiert:
\begin{equation*}
    h_\theta(x) = \frac{1}{1 + e^{-(\theta_0 + \theta_1 x_1 + \dots + \theta_n x_n)}}
\end{equation*}
L ist die Kostenfunktion und definiert als:
\begin{equation*}
    L_\lambda(D, h_\theta)=L(D,h_\theta)+\lambda R(h_\theta)
\end{equation*}
Mit
\begin{equation*}
    L(D,h_\theta)=- \sum_{i=1}^{m}  y_i \ln h_\theta(x_i) + (1 - y_i) \ln (1 - h_\theta(x_i))
\end{equation*}
und den Regularisierungsmethoden $R(h_\theta)=\sum_{i=1}^m\theta^2$ (Ridge-Regression) und $ R(h_\theta)=\sum_{i=1}^m\|\theta\|$ (Lasso-Regression).
 



\subsection{Convolutional Neural Network}
\label{CNN}
\subsubsection{Datenvorverarbeitung}
Für die Datenvorverarbeitung der Wikipedia-Artikel wird zunächst ein Byte-Pair-Encoding-Algorithmus (BPE) - genauer der Tokenizer \texttt{cl100k\_base} aus der \texttt{tiktoken}-Bibliothek - angewendet. Dabei werden die Rohtexte in eine Sequenz von numerischen Token umgewandelt, die als Grundlage für die weitere Verarbeitung dienen. Anschließend wird der tokenisierte Text in einen Tensor konvertiert. Um eine einheitliche Eingabelänge zu gewährleisten, werden die Sequenzen auf eine fest definierte maximale Länge normiert: Kürzere Sequenzen werden mit einem speziellen Padding-Token aufgefüllt, während längere Sequenzen abgeschnitten werden. Die resultierenden Daten werden in einem PyTorch Dataset organisiert und mittels eines DataLoaders in Batches aufgeteilt, was eine effiziente Verarbeitung und einen reibungslosen Trainingsablauf ermöglicht.

\subsubsection{Modellarchitektur}
Das eingesetzte Modell basiert auf einem Convolutional Neural Network (CNN). Zunächst wird der tokenisierte Eingabetext über eine Einbettungsschicht (Embedding Layer) geleitet, die die diskreten Token in dichte, kontinuierliche Vektoren umwandelt. Auf den resultierenden Embeddings werden anschließend mehrere 1D-Convolutional Layers mit unterschiedlichen Filtergrößen angewendet. Diese Filter erfassen lokale Muster und n-Gramme im Text, wobei jede Faltungsoperation von einer ReLU-Aktivierungsfunktion gefolgt wird, um nichtlineare Zusammenhänge zu modellieren. Im Anschluss erfolgt ein Global Max-Pooling, das die wichtigsten Merkmale aus den erzeugten Feature-Maps extrahiert. Durch den Einsatz von Dropout wird zudem das Risiko eines Overfittings reduziert, bevor die gewonnenen Merkmale in einem Fully Connected Layer final zu Klassifikationslogits verarbeitet werden.

- Bild von Modellarchitektur? Oder von Faltungsoperation?

\subsubsection{Training}
Für das Training der Modelle kommen unterschiedliche Verlustfunktionen zum Einsatz, abhängig von der spezifischen Aufgabenstellung. Im binären Klassifikationsfall, bei dem entschieden wird, ob ein Artikel neutral oder nicht neutral ist, wird die \texttt{CrossEntropyLoss} verwendet. Im Multilabel-Fall, in dem nicht-neutrale Artikel in mehrere Kategorien wie \textit{fanpov} oder \textit{resume} eingeordnet werden, wird die \texttt{BCEWithLogitsLoss} genutzt, da ein Artikel gleichzeitig mehreren Klassen zugeordnet werden kann. Die Anpassung der Modellparameter erfolgt mithilfe des Adam-Optimierers. Aufgrund des hohen Rechenaufwands und der großen Datenmenge wird das Training auf einer GPU durchgeführt, um die Berechnungen erheblich zu beschleunigen.

\subsubsection{Hyperparameter-Optimierung}
Um die Performance des Modells weiter zu verbessern, wird eine systematische Hyperparameter-Optimierung durchgeführt. Dabei werden Parameter wie die Lernrate, die Anzahl der Filter, die Filtergrößen, die Dropoutrate sowie die maximale Sequenzlänge variiert und optimiert. Mithilfe von Cross-Validation und eines separaten Validierungsdatensatzes wird sichergestellt, dass die gewählten Hyperparameter zu einer guten Generalisierungsfähigkeit des Modells führen.
 
%Als Deep Learning Verfahren wurden Convolutional Neural Networks (CNN) verwendet. Zunächst war die Idee ein Neuronales Netz mit Rückkoppelung zu verwenden. Da das aber zu schlechten Ergebnissen führte wurden versuche mit einem Convolutional Neural Network gestartet. Diese hatten einen grösseren Erfolg, was dazu geführt hat, dass diese weiter als Ansatz verwendet worden sind.

\subsection{Fünfter Ansatz}
\label{Transformer}

\subsubsection{Zweite Variante Beschreibung technischer Details}
Nachdem die ersten vier Ansätze zwar gute Ergebnisse bei der binären Klassifikation gezeigt haben, zeigten sie Schwächen in der Mehrklassenklassifikation. Aus diesem Grund wurden die Daten noch einmal genauer analysiert. Dabei wurden einzelne Artikel noch einmal stichprobenartig gewählt und analysiert. Das Ziel dieser Analyse war es, die Unterschiede der einzelnen Promotional-Klassen zu erkennen. Dabei stellte sich heraus, dass der Hauptunterschied weniger an der Struktur oder Ähnlichem liegt, sondern am Thema des Artikels zu liegen scheint. Aus diesem Grund wurden recherchen gemacht die zur Transformer Architektur, genauer gesagt BERT führten. Im folgenden werden diese Architekturen etwas genauer vorgestellt.
\paragraph{Transformers}
Die Transformer Architektur wurde zum ersten mal von Vaswani et al. \cite{Attention} vorgestellt. Als Eingabe des Transformers werden zu den einzelnen Zeichen Positional Encoding dazugegeben und anschliessend werden sie durch Embeddings kodiert. Der Transformer selbst wird in einen Encoder und einen Decoder unterteilt wird. Sowohl Encoder als auch Decoder bestanden dort aus 6 Schichten. Im Encoder wurden diese 6 Schichten in eine Multihead-Attention Schicht und ein Vorwärtsgerichtetes Netzwerk unterteilt. Die 6 Schichten des Decoders haben ebenfalls diese 2 Unterschichten. Es kommt hier allerdings noch eine weitere Schicht hinzu, die eine Multihead-Attention über die Ausgabe ausführt. Die Attention-Funktionen werden mithilfe des Skalierten Skalarproduktes berechnet. Diesem werden als Eingabe Abrfragen, Schlüssel und Werte übergeben. Diese werden in Form der Matrizen $Q$ für die Abfragen, $V$ für die Werte und $K$ für die Schlüssel übergeben. Das Skalarprodukt berechnet sich dann wie folgt:
$${Attention(Q,K,V)} = {\frac{QK^T}{\sqrt{d_k}} V}$$
Es werden mehrere Attentionfunktionen über Teilmengen der Eingabedaten gleichzeitig berechnet und anschliessend Concateniert. Dieses Konzept führt zur Multi-head attention. Die Attention-funktion kommt in drei Punkten zum Vorschein. Die Encoder-Decoder Attention sorgt, dass der Decoder auf alle Ausgaben des Encoders zugreifen kann. Die Encoder Attention sorgt, dass der Encoder auf alle seine vorherigen Schichten zugreifen kann und der Die Decoder Self Attention sorgt, dass der Decoder nur auf seine Vorherigen Schichten zugreifen kann.
\paragraph{Bidirectional Encoder Representations from Transformers}
Der Bidirectional Encoder Representations from Transformers (BERT) wurde von Devlin et al.  \cite{BERTReference} vorgestellt. BERT wurde so entwickelt, dass es auf einem riesigen nicht-gelabelten Datensatz trainiert wird. Das erlaubt dem Modell mithilfe einer weiteren Schicht auf eine grosse Anzahl Aufgaben abgestimmt zu sein. Während die ursprüngliche Variante von Transformern nur die vorherigen Wörter in die Berechnung einbezieht, betrachtet BERT auch die Wörter die nach dem Wort stehen. Dadurch bezieht es Kontext von beiden Seiten mit ein. BERT arbeitet, indem es manche Wörter maskiert und versucht diese anhand ihres Kontexts zu bestimmen. BERT wurde zunächst mit unüberwachtem Lernen trainiert und kann anschliessend gefine-tuned werden. Eine der Eigenschaften, die dazu geführt haben, für dieses Praktikum BERT zu verwenden, war dass es auf u.a. auf dem Wikipedia Corpus trainiert worden ist. Dadurch "kennt" es den Aufbau der Artikel bereits und musste nur noch gefine-tuned werden, sodass es Daten klassifizieren kann. 

%\subsection{Vortrainierte Transformer Modelle}
%Für die verwendete Methode wurden Vortrainierte Transformer Modelle verwendet. Dabei wurden verschiedene Transformer betrachtet. 

%\subsection{Datenaugmentierung}

\subsection{Transfer Learning und Vortrainierte Embeddings}

https://arxiv.org/pdf/1301.3781 Falls wir Word2Vec verwenden.

https://nlp.stanford.edu/pubs/glove.pdf falls wir GloVe verwenden.


https://arxiv.org/pdf/1607.04606 falls wir Fast Text verwenden.

https://arxiv.org/pdf/1802.05365 Falls wir Elmo verwenden wollen. Ich finde das sehr interessant.


https://arxiv.org/pdf/1810.04805 Falls wir Bert verwenden wollen. Berts Vorteile liegen auch stark darin dass es genau auf dem Wikipedia Corpus trainiert worden ist.

 
%https://arxiv.org/pdf/2308.00939

\section{Experimente}
% Text?

\subsection{Projektstruktur}
\label{sec:projektstruktur}
Um eine konsistente Datenverarbeitung und vergleichbare Ergebnisse zu gewährleisten, wurde eine modulare Pipeline in einem Python-Package implementiert. Diese Pipeline umfasst Methoden zum Laden, Vorverarbeiten, Extrahieren von Merkmalen, Trainieren und Evaluieren von Modellen. Alle Komponenten können sowohl in der Pipeline als auch in Jupyter-Notebooks verwendet werden. 

Die Pipeline wird über YAML-Konfigurationsdateien gesteuert, die Parameter wie Dateipfade, aktivierte Vorverarbeitungsschritte, Modellparameter und Evaluationskriterien enthalten. Zudem erlaubt sie die partielle Ausführung einzelner Schritte, sodass beispielsweise nur das Modell mit variierenden Parametern trainiert werden kann, was die Laufzeit reduziert.

Die Modelle sind als Klassen implementiert, die von einer abstrakten Basisklasse erben, um eine einheitliche Schnittstelle für die Pipeline bereitzustellen. Die klassischen maschinellen Lernmethoden lassen sich direkt über die Konfigurationsdateien auswählen. Aufgrund zusätzlicher Komplexität wurden die Deep Learning Ansätze nicht in die Pipeline integriert, sie verwenden jedoch Methoden aus der Pipeline.

\subsection{Evaluationsmetriken}
\label{sec:evaluationsmetriken}
Sei $D$ ein Datensatz. In der binären Klassifikation liegt unser Hauptfokus primär auf der Optimierung des \textbf{Recalls} $\rec(D, \clf)$ und sekundär auf der \textbf{Precision} $\prec(D, \clf)$. Dabei ist $\clf\colon\mathbb{R}^d\to \{0, 1\}$ ein binärer Klassifikator.\\

Für die Multilabel-Klassifikation betrachten wir den Klassifikator $\clf\colon\mathbb{R}^d\to\{0, 1\}^k$, wobei $k$ die Anzahl der verschiedenen Labels ist (in diesem Bericht $k=5$). Komponentenweise können wir $\clf$ auch schreiben als $\clf = (\clf_i)_{1\leq i\leq k}$, wobei $\clf_i\colon\mathbb{R}^d\to\{0, 1\}$ ein binärer Klassifikator ist. Wir definieren den \textbf{Macro Average Recall} von $\clf$ als
\begin{equation*}
    \operatorname{macro\,avg\,rec}(D, \clf) = \frac{1}{k}\sum_{i=1}^k\rec (D, \clf_i).
\end{equation*}

\subsection{Ergebnisse}
\begin{center}
    \begin{tabular}{|c|c|c|c|c|}
        \hline
        Ansatz                       & acc  & prec & rec  & F1   \\
        \hline
        Logistische Regression       & 0.99 & 0.98 & 0.92 & 0.93 \\
        Bayes-Klassifikator          & 0.95 & 0.98 & 0.93 & 0.91 \\
        Support Vector Machine       & 0.94 & 0.92 & 0.89 & 0.95 \\
        Convolutional Neural Network & 0.91 & 0.94 & 0.92 & 0.96 \\
        Transformer                  & 0.97 & 0.95 & 0.92 & 0.93 \\
        \hline
    \end{tabular}
\end{center}

\label{Experimente}

\section{Ausblick}
\label{Ausblick}
Nachdem alle Experimente durchgeführt und evaluiert worden sind, werden die gesammelten Erfahrungen festgehalten.

\subsection{Erfolglose Versuche}
Um gute Modelle zu finden, die die Aufgabe lösen konnten, gab es einige Versuche, die zwar vielversprechend schienen, aber in der Praxis keine erfolgreichen Ergebnisse lieferten.

\subsubsection{Gram-Schmidt Verfahren zum Imitieren von Embeddings}
Embeddings wurden in \textbf{Referenz ergänzen} vorgestellt. Ein Problem, die Embeddings haben, ist dass sie vortrainiert werden müssen. Ein Gedanke, der bereits früh in der Entwicklung der Modelle auftrat, war es Embeddings zu imitieren und sie so zu erschaffen, dass es nicht notwendig ist, diese zu trainieren. Yang et al. \cite{Yang2019} hatte eine ähnliche Idee. Die Grundidee ist es, die Sätze aufzuspannen und aufgrund dessen die Ähnlichkeit zu einander zu bestimmen. Im Rahmen des Projektpraktikums wurden daher $n$ Artikel von jeder der $m$ Kategorien verwendet, um eine Orthonormalbasis für einen $m \cdot n$-dimensionalen Raum zu schaffen. Daraufhin wurden die Skalarprodukte zwischen den restlichen Vektoren und jedem Vektor der Orthogonalbasis bestimmt. Anschließend wurde das Skalarprodukt der Vektoren der nicht aufgespannten Vektoren mit dem der orthonormierten Vektoren gebildet. Diese Skalarprodukte, waren jeweils der $x_i$ Koordinatenpunkt, wobei $i$ die Koordinate ist, die durch $n$ aufgespannt wird. Diese wurden anschließend als Eingabe der anderen Modelle verwendet. Das Problem an dieser Lösung war, dass die Resultate schlechter geworden sind, weswegen diese Lösung nicht weiter verfolgt worden ist.

\subsubsection{Klassifizierung einzelner Wörter als Kodierung}
Da die vier ersten verwendeten Ansätze zu Problemen führten, was die Multiklassenklassifikation angeht und daher der Gedanke aufkam den Kontext von Wörtern miteinzubeziehen, gab es den Gedanken, bereits einzelne Wörter zu klassifizieren und anschließend so zu kodieren, dass ähnlich klassifizierte Wörter durch naheliegende Werte dargestellt werden. Dabei wurde das Alphabet als Lexikon verwendet und die Wörter mithilfe dieses Lexikons kodiert. Diese Kodierung wurde dann klassifiziert. Das Problem an diesem Versuch, war es dass bereits die Klassifizierung der einzelnen Wörter zu schlechten Resultaten führen geführt hat, weswegen dieser Versuch an dieser Stelle nicht weiter verfolgt worden ist. Das Verfahren funktioniert grundsätzlich wie folgt: Zunächst werden alle Artikel in ihre Worte zerteilt. Jedes der Wörter wird aufgrund seiner Angehörigkeit gelabelt. Anschließend wird geschaut, ob ein Wort mehrere Label hat. Dieses wird entsprechend separat gelabelt. Bisher wurden nur die Wörter aus den Trainingsdaten gelabelt. Diese Wörter werden in ihre Buchstaben zerlegt und faktorisiert. Anschließend werden sie verwendet, um ein Modell zu trainieren. Neue Wörter, das heißt z.B. Wörter aus den Testdaten, die nicht in den Trainingsdaten waren, werden in einzelne Buchstaben zerlegt und vektorisiert. Daraufhin werden sie klassifiziert und erhalten ihren Code, mit dem sie als Vektor dargestellt werden.

\subsection{Ausbaumöglichkeiten}
Um die Forschung auf diesem Gebiet fortzuführen, könnte man die Modelle auf dem gesamten Wikipedia Dump trainieren. Ausserdem könnte man die Arbeit ausbauen, indem man weitere Modelle hinzufügt. Besonders interessant könnten dabei parameterlose Modelle wie Entscheidungsbäume sein, um die Multilabelklassifikation weiter zu verbessern. Eine weitere Architektur, die positive Effekte zur Bekämpfung der ungleich verteilten Label haben könnte, wäre SetFit \cite{Tunstall2022}. Im Rahmen des Praktikums, lag der Fokus eher auf dem Text der einzelnen Artikel, während die URL nicht zur anwendung kam. Daher könnte man versuchen, in einem weiteren Schritt eher auf dieses Attribut zu fokussieren und die Metadaten der einzelnen Seiten zur Klassifizierung zu verwenden.

\label{Ausblick}


\section{Zusammenfassung und Fazit}
\label{ZusammenfassungUndFazit}
Das Ziel dieser Arbeit war die Untersuchung von Modellen zur Analyse der Qualität von Wikipediaartikeln. Dabei wurde zunächst der gegebene Datensatz analysiert und Problemstellungen erarbeitet. Dieser Datensatz wrude mithilfe des Wikipedia Dumps erweitert. Weitere Versuche der Datenaugmentierung haben stattgefunden. Es wurden im Rahmen des Projekts fünf Modelle implementiert und ausgewertet. Anschliessend wurden die Ergebnisse verglichen und evaluiert. Artefakte, welche durch die Arbeit entstanden, waren eine Pipeline, welche die 3 klassischen Modelle implementiert und eine flexible Verwendung erlaubt, wobei verschiedene Vorverarbeitungsmethoden und Vektorisierungen verwendet werden können. Getrennt davon wurden die Deep Learning Ansätze erarbeitet und können z.T. auch flexibel verwendet werden. Ausserdem entstand durch das Praktikum ein riesiger Wikipedia Dump Datensatz, welcher 56 GB Daten zur Verfügung stellt.
Der Schluss aus der Auswertung der Resultate war, dass die Binäre und Dreiklassenklassifikation sehr gute Resultate erzielt haben. Dabei schnitten SVM, KNN und logistische Regression mit 96 Prozent Recall bei der binären Klassifikation und SVM und logistische Regression mit 90 Prozent bei der Dreiklassenklassifikation am Besten ab. Die logistische Regression ist das sinnvollste Modell, weil es schneller als das SVM trainiert wird. Bei der Multilabelklassifizierung wurden weniger gute Resultate erzielt. Dort schnitt der Bayes Klassifikator mit 66 Prozentiger Sensitivtät am Besten ab. Um die Forschung fortzusetzen, könnte man weitere Modelle implementieren. Dabei könnte man z.B. SetFit verwenden, welches zur Bekämpfung der unterrepräsentierten Klassen helfen kann. Dieses Problem könnte auch durch das Hinzuziehen weiterer Daten abgeschwächt werden. Ausserdem könnte man die Forschung auf andere Artikel als nur Wikipediaartikel ausweiten. Dabei könnte man die trainierten Modelle auf z.B. wissenschaftlichen Artikeln evaluieren. 

% References
\addreferences

\makestatement{5}

\end{document}