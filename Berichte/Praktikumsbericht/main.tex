\documentclass[researchlab,palatino]{AIGpaper}
% Please read the README.md file for additional information on the parameters and overall usage of AIGpaper

%%%% Package Imports %%%%%%%%%%%%%%%%%%%%%%%%%%%%%%%%%%%%%%%%%%%%%%%%%%%%%%%%%%%%%%%%%
\usepackage{graphicx}					    % enhanced support for graphics
\usepackage{tabularx}				      	% more flexible tabular
\usepackage{amsfonts}					    % math fonts
\usepackage{amssymb}					    % math symbols
\usepackage{amsmath}                        % overall enhancements to math environment
\usepackage{amsthm}                         % Nutzung von Definition
\usepackage{hyperref}                       % Zeilenumbruch in URL
\usepackage{xurl}                           % Nach hyperref laden

%%%% optional packages
\usepackage{tikz}                           % creating graphs and other structures
\usetikzlibrary{arrows,positioning}
\tikzset{
    %Define standard arrow tip
    >=stealth',
    %Define style for argument
    args/.style={circle, minimum size=0.9cm,draw=black, thick,fill=white},
}


%%%% Author and Title Information %%%%%%%%%%%%%%%%%%%%%%%%%%%%%%%%%%%%%%%%%%%%%%%%%%%
\author{Robin Suxdorf \and Sebastian Bunge \and Johannes Krämer \and Emmanuelle Steenhof \and Alexander Kunze}

\title{Web Science - Die Qualität von Wikipedia-Artikeln}


%%%% Abstract %%%%%%%%%%%%%%%%%%%%%%%%%%%%%%%%%%%%%%%%%%%%%%%%%%%%%%%%%%%%%%%%%%%%%%

\germanabstract{
Erstmal nur ein Draft. Inhalte werden weiter abgestimmt. 
}

% use this if the document is written in english
%\englishabstract{}


\begin{document}

\maketitle % prints title and author information, as well as the abstract 


% ===================== Beginning of the actual text section =====================

\section{Einleitung}

Das Ziel dieses Projektpraktikums ist die praktische Anwendung von Methoden des maschinellen Lernens auf einen vorgegebenen Datensatz aus dem Bereich der Web Science. Wir haben uns für das Thema \textbf{Qualität von Wikipedia-Artikeln} entschieden und nutzen dafür den Datensatz von Kaggle: \url{https://www.kaggle.com/datasets/urbanbricks/wikipedia-promotional-articles}\\
Im Rahmen dieses Projekts bearbeiten wir folgende Teilaufgaben:

\begin{enumerate} 
\item \textbf{Analyse des Datensatzes und Identifizierung einer geeigneten Problemstellung}: Wir untersuchen den bereitgestellten Datensatz eingehend, um ein maschinelles Lernproblem zu formulieren, das mit den vorhandenen Daten gelöst werden kann. 
\item \textbf{Aufbereitung und Vorverarbeitung des Datensatzes}: Wir bereinigen und transformieren die Daten, um sie für die Modellierung vorzubereiten. 
\item \textbf{Anwendung von drei klassischen Methoden des maschinellen Lernens}: Basierend auf den Inhalten der Kapitel 2 und 3 des Kurses \glqq Einführung in Maschinelles Lernen\grqq{} implementieren wir drei klassische Algorithmen, um die identifizierte Problemstellung zu adressieren. 
\item \textbf{Anwendung eines Deep-Learning-Ansatzes}: Wir recherchieren einen geeigneten Deep-Learning-Ansatz, setzen diesen um und wenden diesen auf die Problemstellung an
\item \textbf{Entwickeln eines eigenen Ansatzes}: Im Rahmen dieser Ausarbeitung wird ein eigener Ansatz für die Problemstellung entwickelt und beschrieben. 
\item \textbf{Interpretation und Diskussion der Ergebnisse}: Basierend auf den bisherigen Resultaten entwickeln wir eine neue Idee für einen passenden Ansatz, beispielsweise eine neue Architektur für ein neuronales Netzwerk, die wir implementieren und anwenden. \end{enumerate}

\section{Aufgabenverteilung}
Im Kick-Off Meeting wurde Robin Suxdorf als Teamleiter und Kommunikationskanal zu den Praktikumsbetreuern gewählt.  Für die Umsetzung der Teilaufgaben wurden jeweils verantwortliche bestimmt:
\begin{enumerate}
    \item Klassische Methode 1: Ansatz: Bayes Leiter: Sebastian Bunge
    \item Klassische Methode 2: Ansatz: SVM Leiter: Johannes Krämer
    \item Klassische Methode 3: Ansatz: Logistische Regression Leiter: Alexander Kunte
    \item Deep-Learning Methode: LSTM Transformer oder Ansatz über Embeddings; Zuerst einmal werden Embeddings angeschaut Leiter Robin Suxdorf
    \item Eigener Ansatz: wird noch genauer angeschaut Leiter: Emmanuelle Steenhof
\end{enumerate}
Während des gesamten Praktikums schreibt Alexander Kunze fortlaufend den Praktikumsbericht weiter.
-Weitere Themen: Präsentation, Vorträge, usw. 
\section{Teaminterne Organisation}
Im Rahmen des Kick-Offs wurde beschlossen, dass Discord (bereitgestellt über Alexander Kunze und Github (bereitgestellt von Robin Suxdorf) als Kollaborationsplattformen dienen. Ein wöchentlicher Jour-Fixe sichert den regelmäßigen Austausch. Jeder Teilnehmer verantwortet die Weiterentwicklung seiner Methode. Das bedeutet, er entwickelt die Methode weiter, gibt zum Jour-Fixe ein Update zum Stand und teilt mit, wenn es Herausforderungen gibt. Das Team unterstützt dabei jeden Leiter und gibt Feedback bei jeder Statusvorstellung. 

\section{Datensatz und Problemstellung}

\subsection{Datensatz}
Die Datensätze stammen von Kaggle: \url{https://www.kaggle.com/datasets/urbanbricks/wikipedia-promotional-articles}. Ein Datensatz enthält Wikipedia-Artikel, die als \emph{promotional} (also werbend) klassifiziert sind. Dabei sind folgende Label vergeben:
\begin{itemize}
    \item advert – „Dieser Artikel enthält Inhalte, die wie eine Werbeanzeige verfasst sind.“
\item coi – „Ein Hauptautor dieses Artikels scheint eine enge Verbindung zu seinem Thema zu haben.“
\item fanpov – „Dieser Artikel ist möglicherweise aus der Sicht eines Fans geschrieben, statt aus einer neutralen Perspektive.“
\item pr – „Dieser Artikel liest sich wie eine Pressemitteilung oder ein Nachrichtenartikel oder basiert weitgehend auf routinemäßiger Berichterstattung oder Sensationslust.“
\item resume – „Dieser biografische Artikel ist wie ein Lebenslauf geschrieben.“
\end{itemize}
Der zweite Datensatz enthält Wikipedia Artikel die \emph{nicht-promotional} klassifiziert sind.

\subsection{Problemdefinition}

Das Ziel dieses Projekts ist die Entwicklung von Modellen zur automatisierten Klassifikation von Wikipedia-Artikeln als \emph{promotional} (werblich) oder \emph{nicht-promotional}. Wikipedia strebt nach objektiven und neutralen Inhalten; daher ist die Identifizierung von Artikeln mit werbenden Charakter von großer Bedeutung, um die sachliche Qualität der Plattform zu gewährleisten.

\subsection{Zielsetzung}

Die Hauptziele des Projekts sind:

\begin{itemize} \item Entwicklung von drei klassischen maschinellen Lernmodellen und einem Deep-Learning-Modell zur Klassifikation von Wikipedia-Artikeln. \item Vergleich der Modelle anhand von Leistungsmetriken wie Genauigkeit, Präzision, Recall und F1-Score. \item Identifikation des Modells mit der besten Leistung für die gegebene Aufgabe. \end{itemize}

\section{Ansätze}
\subsection{Logistische Regression}
\subsubsection{Binäre Klassifikation}
Um Artikel als \textit{„good“} oder \textit{„promotional“} zu kategorisieren, wurde eine logistische Regression mit L1-Regularisierung eingesetzt, um eine robuste Modellierung und Merkmalsselektion zu ermöglichen. Zunächst wurde der gegebene Datensatz gemäß der eingesetzten Pipeline vorverarbeitet. Zur numerischen Darstellung der Texte wurde eine \textit{TF-IDF-Vektorisierung} angewandt. Hierbei wurden verschiedene Konfigurationen getestet, darunter unterschiedliche \textit{ngram}-Bereiche und eine Begrenzung der maximalen Merkmalsanzahl. Die Daten wurden anschließend in Trainings- und Testmengen aufgeteilt.\\
Das Modell wurde mit einer logistischen Regression mit L1-Regularisierung trainiert. Um die optimale Hyperparameter-Kombination zu finden, wurde ein \textit{GridSearchCV}-Verfahren mit Kreuzvalidierung eingesetzt. Hierbei wurden verschiedene Werte für den Regularisierungsparameter \(C\) sowie unterschiedliche \textit{ngram}-Bereiche getestet. \\
Das trainierte Modell wurde mit Metriken wie \textit{Precision}, \textit{Recall} und \textit{F1-Score} bewertet. Um weitere Einsichten in die Artikelstruktur zu gewinnen, wurden zusätzliche Analysen durchgeführt, darunter eine Untersuchung der Häufigkeit von Sub-Labels sowie eine statistische Analyse der Wortanzahl in den Artikeln.\\
Die Ergebnisse zeigten, dass der Ansatz zuverlässig zwischen \textit{good}- und \textit{promotional}-Artikeln unterscheiden konnte. Zudem konnten durch die L1-Regularisierung irrelevante Merkmale entfernt werden, wodurch das Modell interpretierbarer wurde.
\subsubsection{Erweiterung durch Multilabel-Klassifikation und Datenaugmentation}
Zusätzlich zur binären Klassifikation wurde das Problem als \textit{Multilabel-Klassifikation} betrachtet. Dabei wurde jedem Artikel eine oder mehrere Kategorien aus der Menge \textit{advert}, \textit{coi}, \textit{fanpov}, \textit{pr} und \textit{resume} zugewiesen. \\
Zunächst wurde der Datensatz geladen und mittels \textit{TF-IDF-Vektorisierung} in numerische Features umgewandelt. Da ein Artikel mehrere Labels gleichzeitig besitzen kann, wurde ein \textit{One-vs-Rest}-Ansatz mit L1-regularisierter logistischer Regression verwendet. Das Modell wurde mit einer \textit{Train-Test-Split}-Aufteilung trainiert und mit \textit{Precision}, \textit{Recall} und \textit{F1-Score} bewertet.\\
Da einige Klassen unterrepräsentiert waren, wurde eine Datenaugmentation durchgeführt. Dabei wurden für Artikel mit seltenen Labels durch eine \textit{Synonym-Ersetzung} neue Sätze generiert. Hierbei wurde für zufällig ausgewählte Wörter in den Texten ein Synonym aus dem \textit{WordNet}-Lexikon ersetzt. Diese augmentierten Texte wurden in den Datensatz integriert, um die Klassifikationsleistung für unterrepräsentierte Labels zu verbessern.\\
Nach der Augmentierung wurde das Modell erneut trainiert und evaluiert. Der Vergleich der Klassifikationsberichte zeigte, dass die Verwendung der erweiterten Datenbasis zu einer Erhöhung der \textit{F1-Score} führte.\\
Die durchgeführten Schritte führten zum Teil zu einer verbesserten Klassifikationsleistung, insbesondere für seltene Labels. Das Modell wurde anschließend in die Pipeline des Projektes eingefügt.


\section{Experimente}
Blubb
\subsection{Evaluationsmetriken}
\begin{enumerate}
    \item 
Sei $D = \{(x^{(1)}, y^{(1)}), \dots, (x^{(m)}, y^{(m)})\}$ ein Datensatz und $clf: \mathbb{R}^n \to \{0, 1\}$ ein (binärer) Klassifikator. Das \textbf{Genauigkeitsmaß} $acc$ von $clf$ bezüglich $D$ ist definiert durch
\begin{equation}
    acc(D, clf) = \frac{1}{m} \sum_{i=1}^{m} \left(1 - \left|y^{(i)} - clf(x^{(i)})\right|\right)
\end{equation}
    
\item 
Wir definieren

\begin{equation}
    TP(D, clf) = |\{i \mid y^{(i)} = 1, clf(x^{(i)}) = 1\}|
\end{equation}
\begin{equation}
    TN(D, clf) = |\{i \mid y^{(i)} = 0, clf(x^{(i)}) = 0\}|
\end{equation}
\begin{equation}
    FP(D, clf) = |\{i \mid y^{(i)} = 0, clf(x^{(i)}) = 1\}|
\end{equation}
\begin{equation}
    FN(D, clf) = |\{i \mid y^{(i)} = 1, clf(x^{(i)}) = 0\}|
\end{equation}

Die \textit{Konfusionsmatrix} von $clf$ bzgl. $D$ stellt die vier oben genannten Werte tabellarisch wie folgt dar:

\[
\begin{array}{|c|c|c|}
\hline
 & y = 1 & y = 0 \\
\hline
clf = 1 & TP(D, clf) & FP(D, clf) \\
clf = 0 & FN(D, clf) & TN(D, clf) \\
\hline
\end{array}
\]
\item
Sei $D = \{(x^{(1)}, y^{(1)}), \dots, (x^{(m)}, y^{(m)})\}$ ein Datensatz und $clf: \mathbb{R}^n \to \{0, 1\}$ ein ~(binärer) Klassifikator. Definiere
\begin{itemize}
    \item \textbf{Präzision:}
    \begin{equation}
    \text{prec}(D, clf) = \frac{TP(D, clf)}{TP(D, clf) + FP(D, clf)}
     \end{equation}
    \item \textbf{Recall:}
 \begin{equation}
    \text{rec}(D, clf) = \frac{TP(D, clf)}{TP(D, clf) + FN(D, clf)}
    \end{equation}
    \item \textbf{F1:}
    \begin{equation}
    \text{F1}(D, clf) = \frac{2 \cdot \text{prec}(D, clf) \cdot \text{rec}(D, clf)}{\text{prec}(D, clf) + \text{rec}(D, clf)}
    \end{equation}
\end{itemize}
\end{enumerate}

\section{Ausblick}
Blubb

\section{Zusammenfassung und Fazit}
Blubb
% References
\addreferences

\makestatement{5}

\end{document}